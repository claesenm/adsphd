\chapter{Introduction}\label{ch:introduction}

In this work we explored the ability of screening for (type 2) diabetes mellitus based on Belgian health expenditure data from a machine learning perspective. The thesis is organized as a collection of papers concerning various aspects related to this application.\footnote{Some chapters have minor differences compared to the papers on which they are based.} This chapter provides some context of the project, including the medical background of diabetes and its treatment, a summary of the Belgian health insurance system and this project's data analysis challenges. 

This work aligns with the overall trend towards eHealth and the rising use of all sorts of data for evidence-based medicine. Our work is the first data-driven medical application based on health expenditure records and demonstrates the potential use of this data in clinical applications that might improve patient outcomes while simultaneously reducing healthcare costs. Building useful medical applications based on health expenditure data can be considered a modern interpretation of one of the key missions of Belgian mutual health insurers as defined by law, namely to promote the physical, psychological and social well-being of their members \citep{ziekenfondswet}.

To meet our objective we tackled various machine learning challenges, of both theoretical and practical nature. This work contains contributions to machine learning, including open-source implementations of all proposed methods, and a detailed description of our steps towards building a reliable screening approach for type 2 diabetes that can be applied on a population-wide scale.

We will first briefly introduce diabetes mellitus in Section~\ref{intro:diabetes}. We proceed to describe early detection and intervention of type 2 diabetes (T2D) -- the most common variant of diabetes mellitus -- in Section~\ref{intro:screening}. Next, the Belgian health insurance landscape is described in Section~\ref{intro:health-insurance}. Subsequently, Section~\ref{intro:machine-learning} contains a discussion of the main challenges of this project from a machine learning perspective to explain the necessity of each aspect of our research. Finally, Section~\ref{intro:structure} summarizes the structure of the text and indicates how all chapters (papers) are related to each other.

%\instructionsintroduction

%%%%%%%%%%%%%%%%%%%%%%%%%%%%%%%%%%%%%%%%%%%%%%%%%%%
%               DIABETES
%%%%%%%%%%%%%%%%%%%%%%%%%%%%%%%%%%%%%%%%%%%%%%%%%%%

\section{Diabetes mellitus} 
\label{intro:diabetes}
Diabetes mellitus is a metabolic disorder characterized by chronic hyperglycemia, which is primarily caused by insufficient insulin secretion and/or insulin resistance \citep{alberti1998definition}. The worldwide incidence of diabetes has increased dramatically over the last century due to changes in human behaviour and lifestyle \citep{zimmet2001global, chen2012worldwide}. Diabetes is one of the main threats to human health world wide \citep{king1998global, zimmet2001global, zimmet2000globalization} and is projected to be the 7th leading cause of death by 2030 \citep{mathers2006projections}. Some key facts related to diabetes mellitus are summarized in the diabetes atlas made by the International Diabetes Federation (IDF) (Figure~\ref{intro:diabetes-infogram}).

An overview of the medical background of diabetes mellitus is described in Appendices~\ref{intro:regulation} to \ref{intro:treatment}, which discuss various aspects of the disease, namely the underlying biological problem, common complications and comorbidities, a classification of diabetes based on etiology, the prevalence and burden of the disorder and finally treatment approaches with an emphasis on pharmacotherapy. 
We will focus on type 2 diabetes (T2D) in the remainder of the text. T2D accounts for about $90\%$ of diabetes mellitus patients, but early detection of this highly prevalent disease poses significant challenges to contemporary medicine.

\begin{figure}[p]
  \centering
  \includegraphics[width=\textwidth]{infogram-cropped.pdf}
  \caption{Diabetes atlas released by the International Diabetes Federation (IDF). Reuse of this material was granted by the IDF. The original is available at \url{http://www.idf.org/worlddiabetesday/toolkit/gp/facts-figures}.} 
  \label{intro:diabetes-infogram}
\end{figure}

\subsection{Secondary prevention of type 2 diabetes} \label{intro:screening}
%\paragraph{Secondary prevention of T2D is effective} 
Studies have convincingly shown that early detection and treatment of T2D can prevent or delay complications of the disease \citep{haffner1990cardiovascular,engelgau2000screening, genuth2003implications, holman200810, gaede2008effect, echouffo2011screening}. Additionally, treatment of early-stage T2D is often relatively simple and cheap (e.g. lifestyle changes) compared to the treatment of progressed T2D, which typically involves strict pharmacological therapy along with the treatment of potential complications \citep{pan1997effects,tuomilehto2001prevention,diabetes2002reduction,zammitt2005hypoglycemia}, indicating the value of secondary prevention.

%\paragraph{T2D is often diagnosed very late} 
Despite its widely-recognized importance, secondary prevention through screening for T2D proves to be problematic, as one fourth up to one third of T2D patients are estimated to be undiagnosed in developed countries \citep{diabetesliga, beagley2014global, american2014standards} and typically years pass between the onset of T2D and its clinical diagnosis \citep{harris1992onset}. In fact, the clinical diagnosis of T2D often follows signs of serious complications, which have developed during the latent stage of the disease \citep{rajala1998prevalence,harris2000early, hu2002elevated, american2014standards}. 

%\paragraph{Barriers for early detection of T2D} 
Diagnostic inertia for T2D arises in several ways. First, the disease may remain asymptomatic for many years \citep{alberti1998report}, during which unmanaged hyperglycemia may induce serious and irreversible development of micro -and macrovascular complications \citep{fowler2008microvascular, beagley2014global}. Second, healthcare information related to a specific patient is often fragmented across individual caregivers' databases, potentially causing situations in which various subtle symptoms of diabetes are presented to multiple caregivers, but the diagnosis remains elusive because each individual caregiver cannot see the big picture. Finally, universal screening for T2D is cost-prohibitive \citep{wareham2001should, engelgau2000screening}, though many organizations advise opportunistic screening of high-risk subgroups \citep{world1994prevention, alberti1998report, engelgau2000screening,american2014standards}.

In the remainder of this Section we will discuss existing screening programmes and the Belgian situation and recommendations.

%%%%%%%%%%%%%%%%%%%%%%%%%%%%%%%%%%%%%%%%%%%%%%%%%5
%
%		INTERNATIONAL
%
%%%%%%%%%%%%%%%%%%%%%%%%%%%%%%%%%%%%%%%%%%%%%%%%%5

\ifx
\subsubsection{International recommendations} \label{intro:screening-recommendations}


\begin{table}[!h]
%\begin{tabular}{p{\textwidth}}
%\toprule
\colorbox{gray!20!white}{\parbox{\textwidth}{
\begin{enumerate}
\item Testing should be considered in all adults who are overweight (BMI $\geq25$ kg/m$^{2*}$) and have additional risk factors:
\begin{itemize}
\item physical inactivity,
\item first-degree relative with diabetes,
\item high-risk race/ethnicity (e.g., African American, Latino, Native American, Asian American, Pacific Islander),
\item women who delivered a baby weighing $>9$ lb or were diagnosed with GDM,
\item hypertension ($\geq140/90$ mmHg or on therapy for hypertension),
\item HDL cholesterol level $<35$ mg/dL (0.90 mmol/L) and/or a triglyceride level $>250$ mg/dL (2.82 mmol/L),
\item women with polycystic ovarian syndrome,
\item A1C $\geq5.7\%$, IGT, or IFG on previous testing,
\item other clinical conditions associated with insulin resistance (e.g., severe obesity, acanthosis nigricans)
\item history of CVD.
\end{itemize}
\item In the absence of the above criteria, testing for diabetes should begin at age 45 years.
\item If results are normal, testing should be repeated at least at 3-year intervals, with consideration of more frequent testing depending on initial results (e.g., those with prediabetes should be tested yearly) and risk status.
\item[*] At-risk BMI may be lower in some ethnic groups.
\end{enumerate}
%\bottomrule
%\end{tabular}
}}
\caption{ADA criteria for testing for diabetes in asymptomatic adult individuals, taken from \citep{american2014standards}.}
\label{table:ada-criteria}
\end{table}

\fi

%%%%%%%%%%%%%%%%%%%%%%%%%%%%%%%%%%%%%%%%%%%%%%%%%5
%
%		IN BELGIUM
%
%%%%%%%%%%%%%%%%%%%%%%%%%%%%%%%%%%%%%%%%%%%%%%%%%5

\subsubsection{Existing screening approaches} \label{intro:screening-existing}
The Cambridge Risk Score (CRS) was developed to assess the probability of undiagnosed T2D based on data that is routinely available in primary care records, including age, sex, medication use, family history of diabetes, BMI and smoking status \citep{griffin2000diabetes}, The CRS and comparable scores have been shown to be useful on multiple occasions \citep{baan1999performance,griffin2000diabetes, park2002performance, spijkerman2004performance}. 
The FINDRISC score is based on a 10-year follow-up using age, BMI, waist circumference, history of antihypertensive drugs and high blood glucose, physical activity and diet and is used to predict drug-treated diabetes \citep{lindstrom2003diabetes}. The strongest reported predictors in this study were BMI, waist circumference, history of high blood glucose and physical activity. Gl{\"u}mer et al. \citep{glumer2004danish} developed a risk score based on age, sex, BMI, known hypertension, physical activity and family history of diabetes. The German diabetes risk score is based on age, waist circumference, height, history of hypertension, physical activity, smoking, and diet \citep{schulze2007accurate}.
More complex risk scores include various clinical parameters \citep{heikes2008diabetes, stern2002identification, mcneely2003comparison}.



%Simple questionnaires using only basic information can be a powerful tool to detect people that are at risk for diabetes, as proven in other countries [14–17].
%[14] Jaana Lindstr ̈om and Jaakko Tuomilehto. The diabetes risk score: a practical tool to predict type 2 diabetes risk. Diabetes Care, 26:725–731, 2003.
%[15] Charlotte Glumer, Bendix Carstensen, Annelli Sandbæk, Torsten Lauritzen, Torben Jørgensen, and Knut Borch-Johnsen. A danish diabetes risk score for targeted screening: the inter99 study. Diabetes Care, 27:727–733, 2004.
%[16] Caroline A. Baan, Johannes B. Ruige, Ronald P. Stolk, Jacqueline C.M. Witteman, Jacqueline M. Dekker, Robert J. Heine, and Edith J.M. Feskens. Performance of a predictive model to identify undiagnosed diabetes in a health care setting. Diabetes Care, 22:213–219, 1999.
%[17] Griffin SJ, Little PS, Hales CN, Kinmonth AL, and Wareham NJ. Diabetes risk score: towards earlier detection of type 2 diabetes in general practice. Diabetes Metab Res Rev, 16:164–171, 2000



\subsubsection{Situation in Belgium} \label{intro:screening-belgium}
The IDF estimates over 170,000 undiagnosed diabetes patients in Belgium \citep{IDFatlas}. The Diabetes Liga estimates that currently one out of three T2D patients are undiagnosed and that one out of ten Belgians will have type 2 diabetes in 2030 \citep{diabetesliga}. Domus Medica\footnote{A non-profit organization of general practionners that focuses on preventive medicine.} advises against population-wide screening, though it recommends case finding in high-risk subpopulations \citep{wens2005aanbeveling}, for instance via the risk factors listed in Table~\ref{intro:diabetes-liga-risk}.
\begin{table}[!h]
\colorbox{gray!20!white}{\parbox{\textwidth}{
\begin{itemize}
\item Persons of 18--45 years of age that meet one of the following conditions:
\begin{itemize}
\item prior history of gestational diabetes
\item prior history of stress-induced hyperglycemia
\end{itemize}
\item or two of the following conditions:
\begin{itemize}
\item prior history of giving birth to a baby of over 4.5 kg
\item diabetes in first-line relatives (mother, father, sister, brother)
\item BMI $\geq$ 25 kg/m$^2$
\item waist circumference $>88$ cm (for women) or $>102$ cm (for men)
\item treated for high blood pressure or corticoids
\end{itemize}
\item Persons of 45--64 years of age that meet one of the conditions listed above.
\item Persons above 64 years old, regardless of additional risk factors.
\end{itemize}
}}
\caption{High-risk subpopulations according to the Diabetes Liga \citep{diabetesliga}.} \label{intro:diabetes-liga-risk}
\end{table}

The Belgian Scientific Institute of Public Health (WIV-ISP) reports that screening efforts are increasing in Belgium, but also indicates a need for risk stratification that goes beyond selecting all patients above a given age \citep{wivisp}.




% The onset of type 2 diabetes may occur up to 7 years before clinical diagnosis
% Harris MI, Klein R, Welborn TA, Knuiman MW. Onset of NIDDM occurs at least 4–7 years before clinical diagnosis. Diabetes Care 1992; 15:815–819. Cowie CC, Rus

% see also http://www.nature.com/nature/journal/v414/n6865/full/414782a.html

%The worldwide epidemiology of type 2 diabetes mellitus—present and future perspectives Lei Chen, Dianna J. Magliano and Paul Z. Zimmet
% The causes of the epidemic of T2DM are embedded in a very complex group of genetic and epigenetic systems interacting within an equally complex societal framework that determines behavior and environmental influences. This complexity is reflected in the diverse topics discussed in this Review. In the past few years considerable emphasis has been placed on the effect of the intrauterine environment in the epidemic of T2DM, particularly in the early onset of T2DM and obesity. Prevention of T2DM is a ‘whole-of-life’ task and requires an integrated approach operating from the origin of the disease.

% Full Accounting of Diabetes and Pre-Diabetes in the U.S. Population in 1988 –1994 and 2005–2006
% U.S. was 9.3%, of which 30% was undiagnosed based on fasting plasma glucos

% http://www.diabetesresearchclinicalpractice.com/article/S0168-8227(13)00384-7/abstract
% Globally, 45.8%, or 174.8 million of all diabetes cases in adults are estimated to be undiagnosed, ranging from 24.1% to 75.1% across data regions. An estimated 83.8% of all cases of UDM are in low- and middle-income countries. At a country level, Pacific Island nations have the highest prevalence of UDM.

% http://www.bettycjung.net/Pdfs/Alberti.pdf

% http://europepmc.org/abstract/med/18350480

% http://www.researchgate.net/profile/Leonor_Guariguata/publication/259153000_Global_estimates_of_undiagnosed_diabetes_in_adults_for_2013_for_the_IDF_Diabetes_Atlas/links/54f583fc0cf2ba61506653fb.pdf








%uit \citep{american2014standards}:
%Mass screening of asymptomatic individuals has not effectively identified those with prediabetes or diabetes, and rigorous clinical trials to provide such proof are unlikely to occur. In a large randomized controlled trial (RCT) in Europe, general practice patients between the ages of 40–69 years were screened for diabetes, then randomized by practice to routine diabetes care or intensive treatment of multiple risk factors. After 5.3 years of follow-up, CVD risk factors were modestly but significantly improved with intensive treatment. Incidence of first CVD event and mortality rates were not significantly different between groups \citep{griffin2011effect}. This study would seem to add support for early treatment of screen-detected diabetes, as risk factor control was excellent even in the routine treatment arm and both groups had lower event rates than predicted. The absence of a control unscreened arm limits the ability to definitely prove that screening impacts outcomes. Mathematical modeling studies suggest that screening, independent of risk factors, beginning at age 30 or 45 years is highly cost-effective \$11,000 per quality-adjusted life-year gained) \citep{kahn2010age}.




\section{Belgian mutual health insurance} \label{intro:health-insurance}
The Belgian health care insurance is a broad solidarity-based form of social insurance. Mutual health insurers are the legally-appointed bodies for managing and providing the Belgian compulsory health care and disability insurance. The Belgian sickness fund law of 1990 states that a main goal of mutualities is to promote the physical, psychological and social well-being of their members \citep{ziekenfondswet}.

Joining one of several mutual health insurers or, alternatively, the relief fund for sickness and disability insurance\footnote{In Dutch: Hulpkas voor Ziekte -en Invaliditeitsuitkering (HZIV).} is obliged for anyone who (i) starts working, (ii) is still studying at the age of 25 years or (iii) receives unemployment benefits. Among other things, mutual health insurers are responsible for refunding medical interventions, drug purchases and payments related to disability and pregnancy leave. To implement their operations, mutual health insurers dispose of large databases containing health expenditure records of all their respective members. 

This project was done in close collaboration with the National Alliance of Christian Mutualities (NACM)\footnote{In Dutch: Landsbond der Christelijke Mutualiteiten (LCM).}. NACM  is the largest Belgian mutual health insurer with records of over 4.4 million persons and over 60\% and 40\% market share in Flanders and Belgium, respectively. All data extractions and analyses were done in-house at the department of medical management of NACM in its headquarters in Brussels under supervision of and upon request by the Chief Medical Officer.

We developed a screening system based exclusively on basic personal information (e.g. age, gender) and readily-available health expenditure records collected by NACM, without requiring any external input. The relevant patient-centric information embedded in these records belongs to three key classes:
\begin{itemize}
\item \textbf{Basic biographical information} includes the member's age, gender, place of residence and, if deceased, the date of death. Limited information regarding social status is also available, e.g. whether a member is entitled to increased compensation or suffers from a chronic illness.
\item \textbf{Medical provisions} are encoded via a national nomenclature comprising over 20,000 unique codes. Each medical act yields one or several of these nomenclature codes. 
\item \textbf{Drug purchases} are registered automatically and are encoded per package or per unit when purchased in retail and hospital pharmacies, respectively. In both cases, the encoding contains information about both volume and active substances. 
\end{itemize}

The time-stamped records related to provisions and drug purchases enable constructing a medical resource-use timeline for each patient. As this constitutes the main source of information in our work we will discuss claims records related to provisions and drug purchases in more detail in Sections~\ref{intro:interventions}~and~\ref{intro:drugs}. Finally, we will briefly discuss the overall quality of health expenditure data.


\subsection{Data related to medical interventions} \label{intro:interventions}
Each distinct medical intervention is encoded in a national nomenclature that is maintained by the National Institute for Health and Disability Insurance (NIHDI)\footnote{In Dutch: Rijksinstituut voor ziekte -en invaliditeitsverzekering (RIZIV).} \citep{van2008financing}. After a consultation, patients receive a certificate indicating which provisions were performed (a green, white or blue slip). The patient can then file a claim to get (partially) refunded through his or her mutual health insurer. Refunds can be claimed up to two years after the date of the intervention, though most patients do this more swiftly. In some cases, a copayment system enables the caregiver to get paid directly by the health insurer, removing the need for the patient to claim refunds.

The list of nomenclature numbers can be consulted via the website of the NIHDI and currently comprises over 20,000 unique codes. The sheer amount of codes indicates the fine granularity at which medical interventions are encoded, making it a valuable source of information. Codes can fall out of use when interventions get deprecated or because they get replaced by other codes that are often more specific in some sense.

However, the codes that identify interventions only carry limited information. Specifically, these codes are sufficiently detailed to know which intervention was performed, but do not contain any information regarding its outcome. For example, there are codes indicating blood tests, but the results of these tests are not available to the mutual health insurer. As such, nomenclature codes often serve as proxies for specific diseases, but essentially carry no direct information regarding diagnoses, indications or clinical parameters.

\subsection{Data related to drug purchases} \label{intro:drugs}
Drug purchases work via a copayment system in Belgium, in which the patient only pays his or her share at the time of purchase while the rest is already deducted automatically. As such, drug purchases are automatically recorded and known to health insurers without requiring the patient to explicitly claim refunds and are therefore implicitly complete.

Each drug package carries a \emph{Code Nationale Kode} (CNK) code which indicates the volume in the package and information about the drug itself including its active substances. Hence, these CNK codes carry enough information to map a drug purchase onto one or several codes of the internationally used Anatomical Therapeutic Chemical (ATC) classification system with an associated amount of Defined Daily Doses (DDDs). Tables to map CNK codes onto ATC codes are provided freely by the BCFI\footnote{In Dutch: Belgisch Centrum voor Farmacotherapeutische Informatie.} and the APB\footnote{In Dutch: Algemene Pharmaceutische Bond.}.

The ATC classification system is maintained by the World Health Organization and divides active substances into different groups based on the organ or system on which they act and their therapeutic, pharmacological and chemical properties \citep{world1996guidelines}. Each drug is classified in groups at 5 levels in the ATC hierarchy: fourteen main groups (1st level), pharmacological/therapeutic subgroups (2nd level), chemical subgroups (3rd and 4th level) and the chemical substance (5th level). Figure~\ref{intro:atc-example} illustrates the structure of ATC system for common antidiabetic drugs.

\begin{figure}[!h]
\dirtree{%
.1 \textsc{A}: alimentary tract and metabolism.
.2 \textsc{A10}: drugs used in diabetes.
.3 \textsc{A10A}: {\color{blue}insulins} and analogues.
.3 \textsc{A10B}: blood glucose-lowering drugs, excluding insulins.
.4 \textsc{A10BA}: {\color{blue}biguanides}.
.5 \textsc{A10BA02}: {\color{red}metformin}.
.4 \textsc{A10BB}: {\color{blue}sulfonylureas}.
.5 \textsc{A10BB01}: {\color{red}glibenclamide}.
.5 \textsc{A10BB08}: {\color{red}gliquidone}.
.5 \textsc{A10BB12}: {\color{red}glimepiride}.
.4 \textsc{A10BF}: {\color{blue}alpha-glucosidase inhibitors}.
.5 \textsc{A10BF01}: {\color{red}acarbose}.
.4 \textsc{A10BG}: {\color{blue}thiazolinediones}.
.5 \textsc{A10BG03}: {\color{red}pioglitazone}.
.4 \textsc{A10BH}: {\color{blue}dipeptidyl peptidase 4 (DPP-4) inhibitors}.
.5 \textsc{A10BH01}: {\color{red}sitagliptin}.
.5 \textsc{A10BH02}: {\color{red}vildagliptin}.
}
\caption{Example of the ATC hierarchy for common GLAs. A brief explanation of these different active substances is given in Section~\ref{intro:treatment}.} \label{intro:atc-example}
\end{figure}


\subsection{Quality of health expenditure data}
Overall, health expenditure data can be considered complete, due to the clear financial incentive for patients and caregivers to file claims. The automated registration of drug purchases also contributes to this aspect.

A key benefit of Belgian health expenditure data is that it integrates resource-use from all medical sources. Patients may consult multiple caregivers and institutions but each patient can only be affiliated to one mutual health insurer at a given moment in time. Additionally, most Belgians never switch mutual health insurer.

Health expenditure records give a fine-grained overview of patients' medical histories thanks to the detailed encoding of provisions and drug packages. However, the absence of data related to outcomes, diagnoses and clinical parameters constitutes an important limitation. In this regard, it must be noted that a lot of relevant information to screen for T2D is missing, such as glycated hemoglobin levels, lifestyle, BMI, potential genetic predisposition, \ldots

Health expenditure records prove extremely useful for retrospective observational studies, as is common in epidemiology. However, a certain lag exist between medical acts  and the appearance of associated records in health expenditure databases. For provision records, the maximum lag is two years, while for drug purchases the lag is less than half a year. These lags present problems for applications that require quasi real-time information, such as disease outbreak detection, but are less problematic for screening.

A disadvantage is that expenditure data may be noisy. Most sources of noise can be considered random and hence neglected, but some structural issues exist as well. A specific example is fraud through \emph{upcoding}, which refers to caregivers that wilfully report wrong nomenclature codes, or codes corresponding to provisions that were never performed, in order to obtain higher refunds. This phenomenon is known to plague healthcare systems in various countries and likely occurs in Belgium as well to some extent \citep{silverman2004medicare,steinbusch2007risk,berta2010effects}.

Finally, it is worth noting that the potential of health expenditure data for clinical applications is also being investigated in other countries. A highly visible recent example was the Heritage Health Prize competition to identify patients who will be admitted to a hospital within the next year using historical claims data, with an impressive \$3,000,000 prize pool.\footnote{More information is available at \url{http://www.heritagehealthprize.com/}.}



%%%%%%%%%%%%%%%%%%%%%%%%%%%%%%%%%%%%%%%%%%%%%%%%%%%
%               MACHINE LEARNING
%%%%%%%%%%%%%%%%%%%%%%%%%%%%%%%%%%%%%%%%%%%%%%%%%%%

\section{Machine learning challenges and contributions} \label{intro:machine-learning}
Identifying individuals at high risk for (type 2) diabetes based on Belgian health expenditure data posed several machine learning challenges, including dealing with missing and noisy information, defining representations that capture the implicit structure of health expenditure data and coping with the size of the learning problem, in terms of number of instances and features alike.

This Section highlights the main machine learning contributions made during this project. We will briefly describe the fundamental problems that were tackled, outline the approach and clarify how it fits into the overarching theme of identifying patients at risk for diabetes based on health expenditure data. Chapters~\ref{ch:ensemblesvm}~to~\ref{ch:diabetesjmlr} describe the solutions and methodologies we developed in detail.

We approached the screening task as a binary classification problem. Binary classifiers are models which yield some level of confidence that an instance belongs to the positive class, based on the features of that instance. In our application, every patient represents an instance. Each patient is either diabetic (positive) or non-diabetic (negative). The biographic information and resource-use history of each patient represent the associated features.

The relationships between all aspects of this project's machine learning research are depicted in Figure~\ref{fig:ml-structure}. Our contributions revolved around three focal points: learning from positive and unlabeled data, automated hyperparameter optimization and the development of reusable open-source software to facilitate reproducibility. Sections~\ref{intro:pulearning}~to~\ref{intro:software} describe each focal point in more detail.

\begin{figure}[!h]
\centering
\resizebox{\textwidth}{!}{
\begin{tikzpicture}[mindmap,
  every node/.style={font=\large},
  every concept/.style={minimum size=3cm, text width=3cm, fill=none, very thick},
  level 1 concept/.append style={level distance=150,sibling angle=120}]

  \begin{scope}[mindmap, text=black, minimum size=3cm]
    \node [concept, minimum size=4cm, color=black, text=black] at (0, 0) (diabetesjmlr) {\Large{\textbf{Diabetes Screening Chapter \ref{ch:diabetesjmlr}}}}
	[clockwise from=90, level distance=130] 
	child[concept color=green!50!black] {node [concept, text=black, fill=green!50!black, fill opacity=0.2, text opacity=1] at (-0.5, -0.2) (pulearning) {\Large{\textbf{Semi-supervised Learning}}}
		[clockwise from=0, sibling angle=180]
		child {node [concept] at (1.7, -0.4) (resvm) {Robust PU learning with SVM models \\ Chapter \ref{ch:resvm}}}
		child {node [concept] at (-5.6, 0.7) (evaluation) {Evaluating models without known negatives \\ Chapter \ref{ch:evaluation}}}
	      }
	child[concept color=cyan!80!black] {node [concept, text=black, fill=cyan!80!black, fill opacity=0.2, text opacity=1] at (0.8, -0.2) (software) {\Large{\textbf{Open-Source \\ Software}}}
		[clockwise from=200, level distance=200, sibling angle=90]
		child {node [concept] at (-1.2, -1.5) (optunity) {Optunity \\ Chapter \ref{ch:optunity}}}
		child {node [concept] at (3.2, 2.4) (esvm) {EnsembleSVM \\ Chapter \ref{ch:ensemblesvm}}}
	      }
	child[concept color=red!80!black]{node [concept, text=black, fill=red!80!black, fill opacity=0.2, text opacity=1] at (-1.1, 1.3) (tuning) {\Large{\textbf{Hyper-parameter Search}}}
		[clockwise from=270, level distance=200, sibling angle=90]
		child {node [concept] at (2.4, -1.1) (mic) {Optimization challenges \\ Chapter \ref{ch:mic2015}}}
	      };
  \end{scope}

%\path (resvm) to[circle connection bar switch color=from (cyan!80!black) to (red!80!black)] (esvm) ;
\path (tuning) to[circle connection bar switch color=from (red!80!black) to (cyan!80!black)] (optunity) ;
%\path (mic) to[circle connection bar switch color=from (red!80!black) to (cyan!80!black)] (optunity) ;
\path (tuning) to[circle connection bar switch color=from (red!80!black) to (green!50!black)] (evaluation) ;

\tikzset{>={Latex[width=3mm,length=3mm]}}

\draw[dotted, ->, ultra thick, color=gray, shorten >= 1mm, shorten <= 1mm] (evaluation) to[bend right=11] (optunity) ;
\draw[dotted, ->, ultra thick, color=gray, shorten >= 1mm, shorten <= 1mm] (optunity) to[bend right=5] (resvm) ;
\draw[dotted, ->, ultra thick, color=gray, shorten >= 1mm, shorten <= 1mm] (esvm) to[bend left=30] (resvm) ;
\draw[dotted, ->, ultra thick, color=gray, shorten >= 1mm, shorten <= 1mm] (mic) to[bend left=10] (optunity) ;
\draw[dotted, ->, ultra thick, color=gray, shorten >= 1mm, shorten <= 1mm] (optunity) to[bend left=10] (mic) ;

\draw[dotted, ->, ultra thick, color=gray, shorten >= 1mm, shorten <= 1mm] (resvm) -> (diabetesjmlr) ;
\draw[dotted, ->, ultra thick, color=gray, shorten >= 1mm, shorten <= 1mm] (evaluation) -> (diabetesjmlr) ;
\draw[dotted, ->, ultra thick, color=gray, shorten >= 1mm, shorten <= 1mm] (optunity) -> (diabetesjmlr) ;
\draw[dotted, ->, ultra thick, color=gray, shorten >= 1mm, shorten <= 1mm] (esvm) -> (diabetesjmlr) ;

\end{tikzpicture}
}
\caption{Dependencies of the key contributions made to machine learning research during this project. To enable diabetes screening based on Belgian health expenditure data (Chapter~\ref{ch:diabetesjmlr}), we made contributions to semi-supervised learning (Chapters \ref{ch:resvm} and \ref{ch:evaluation}), automated hyperparameter search (Chapters \ref{ch:mic2015} and \ref{ch:optunity}) and open-source machine learning software (Chapters \ref{ch:ensemblesvm} and \ref{ch:optunity}).} \label{fig:ml-structure}
\end{figure}

\subsection{Learning from positive and unlabeled data} \label{intro:pulearning}
A critical challenge inherent to our application is an infeasibility to ascertain which patients are non-diabetic based only on health expenditure records. This problem originates from several sources, most notably because a significant fraction of diabetic patients is undiagnosed (as discussed in Section~\ref{intro:screening}) and additionally because initial diabetes therapies may exclusively consist of lifestyle changes (cfr. Section~\ref{intro:treatment}) which are not recorded in health expenditure data. 

Fortunately, we were able to identify a reasonable set of known diabetics (positives) based on health expenditure data. Positives were identified via the use of GLA therapy over extended periods of time. The thus identified set of positives mainly consists of patients with progressed diabetes (since the treatment involves pharmacotherapy) and omits patients with (potentially diagnosed) prediabetes. It must be noted that this labeling induces some false positives, mainly due to the use of GLA medication for alternative reasons (e.g. use of metformin for weight loss).

In machine learning, the true class of an instance is commonly called its \emph{label}. Given the labeling issues described previously, we had to learn binary classifiers from positive and unlabeled data to enable screening based exclusively on health expenditure data. This learning scenario is receiving increasing amounts of research attention and is commonly dubbed PU learning. During this project, we improved an existing PU learning method and additionally developed a method to evaluate classifiers without known negatives. The latter aspect is particularly important because it has significant practical implications and was previously uncharted territory.

\subsubsection{Building classifiers with only positive and unlabeled data} 
This is a common topic within semi-supervised learning, presenting additional complexity compared to fully supervised binary classification.\footnote{In this context, fully supervised means all class labels are known.} Various methods have been devised to cope with the increased uncertainty, based on one of two fundamental approaches:
(i) first attempt to infer a set of likely negatives from the unlabeled data and then train a fully supervised model to distinguish known positives from inferred negatives \citep{liu02partially,Yu:2005:SCM:1108759.1108762,Li03learningto} vis-\`a-vis (ii) treat unlabeled instances as negatives with noisy class labels and deal with this directly \citep{Elkan:2008:LCO:1401890.1401920,Lee03learningwith,Liu:2003:BTC:951949.952139,mordelet2014bagging,Liu:2005:PSC:2138033.2138052}.

The method we developed fits into the latter category and is described in detail in Chapter~\ref{ch:resvm}. Our technique achieves state-of-the-art performance in PU learning and is additionally designed for robustness against false positives, which are known to exist in our application. False positives significantly deteriorate the performance of other existing methods, limiting their usability in our project.

\subsubsection{Evaluating classifiers with only positive and unlabeled data} 
Assessing the performance of binary classifiers without known negatives was an open problem at the start of the project. Prior to our work, a few methods have been devised for model selection in PU learning which allow basic pairwise comparisons between classifiers \citep{Lee03learningwith}. During our work, some additional related methods were developed by others \citep{sechidis2014statistical, hajizadeh2014evaluating}. However, none of these quantify the performance of a given classifier in terms of commonly used metrics like sensitivity, specificity and area under the ROC curve.

The performance of models for screening must be quantified before their use can even be considered, however no convincing method to quantify performance without known negatives existed. To circumvent this problem we initially considered manually obtaining negative labels by directly asking patients whether or not they had diabetes. Clearly, this is a very sensitive matter and would additionally have been labour intensive to acquire a sufficient amount of negative labels. 

Instead of manual labeling, we developed a method to reliably estimate performance of binary classifiers without known negatives (cfr. Chapter~\ref{ch:evaluation}). This method is the first of its kind and relies only on the reasonable assumption that known positives are sampled completely at random from all positives, which implies that the distributions of known and latent positives are comparable. Our work effectively reduces estimating performance without negatives to estimating the fraction of positives in the unlabeled set, which is often feasible.

\subsection{Automated hyperparameter optimization} \label{intro:tuning}
Most machine learning algorithms are parameterized, for example to allow the user to determine an optimal model complexity for a given problem. The coefficients of a trained model are commonly called parameters, and hence the parameters used to describe the training problem itself are referred to as \emph{hyper}parameters. Current research focuses on automatically determining suitable values for these hyperparameters \citep{hutter2009paramils, bergstra2011algorithms, snoek2012practical, bergstra2012random, bergstra2013hyperopt, eggensperger2013towards, thornton2013auto}, which essentially boils down to the development of suitable (heuristic) optimizers.

Some of the key challenges in hyperparameter optimization are described in Chapter~\ref{ch:mic2015}. Several libraries have been developed to automate this process and have proven to be far more efficient than manual tuning or grid search \citep{hutter2009paramils, bergstra2012random, snoek2012practical, bergstra2013hyperopt}. However, most of these libraries are challenging to install (even for seasoned programmers!) and feature a lot of complex configurations, hence effectively limiting their potential userbase to experts. To fill this apparant gap of user-friendly tuning software, we have developed a cross-platform open-source Python library that provides a variety of suitable optimizers to automate hyperparameter search via a simple, lightweight API. This library is described in Chapter~\ref{ch:optunity}.


\subsection{Open-source software} \label{intro:software}
Machine learning research requires high-quality, tested and documented software to advance rapidly. Fortunately, several authoraties in the machine learning field are appreciative of open-source software \citep{sonnenburg2007need}. Overall, the field is blessed with a wealth of open-source packages covering all aspects of the learning process and we strongly feel that cultivating this ecosystem is in the best interest of the entire academic community, for reasons such as efficiency, reliability and reproducibility. It is worth noting that every analysis in this project was done using freely available software.

As we recognize the value and importance of a solid open-source ecosystem, we decided to pay it forward by developing two open-source libraries of our own: \emph{EnsembleSVM} and \emph{Optunity}.

\paragraph{EnsembleSVM} is a C$^{++}$ package for ensemble learning with support vector machine (SVM) base models. This software enables efficiently computing nonlinear models on large-scale data sets, which would otherwise be infeasible without significant computational resources. The PU learning method we developed (cfr. Chapter~\ref{ch:resvm}) is a use-case of EnsembleSVM and was implemented entirely using the API offered by the library. EnsembleSVM is described in Chapter~\ref{ch:ensemblesvm}.

\paragraph{Optunity} is a Python library for automated hyperparameter optimization, with interfaces to R, MATLAB, Octave and Java. An overview of Optunity is given in Chapter~\ref{ch:optunity}. At the time of writing, Optunity receives hundreds of downloads each month via the Python Package Index (PyPI). Optunity was used to optimize the hyperparameters of the learning approaches we used to construct models for diabetes screening.


%%%%%%%%%%%%%%%%%%%%%%%%%%%%%%%%%%%%%%%%%%%%%%%%%%%
%               OUTLINE
%%%%%%%%%%%%%%%%%%%%%%%%%%%%%%%%%%%%%%%%%%%%%%%%%%%

\section{Structure of the thesis} \label{intro:structure}
This Section summarizes all subsequent chapters  and reiterates how every aspect is relevant to diabetes screening based on Belgian health expenditure records.

\newcommand{\chapteritem}[2]{\item \emph{Chapter~\ref{ch:#1}} #2}
\begin{itemize}

\chapteritem{survival} describes a study we performed to quantify the survival of Belgian patients after starting various glucose-lowering pharmacotherapies. Unlike other studies, our study does not focus on relative efficacy of different GLA therapies. Instead, it is the only one that provides an expected survival rate for patients starting a specific therapy, accounting for all possible future therapies commonly seen in the Belgian population.

\chapteritem{ensemblesvm}{introduces the EnsembleSVM software package, which provides efficient routines for ensemble learning using SVM base models.}

\chapteritem{resvm}{describes a novel algorithm to learn robust binary classifiers from positive and unlabeled data. The key design criterion is robustness to false positives, which was lacking in existing approaches. The implementation is based on EnsembleSVM (see Chapter~\ref{ch:ensemblesvm}).}

\chapteritem{mic2015}{discusses the main optimization challenges posed by automated hyperparameter search and summarizes the current state-of-the-art.}

\chapteritem{optunity}{describes the Optunity software package, which provides metaheuristic optimization routines for automated hyperparameter optimization. Optunity is available on most commonly used machine learning platforms and tackles the challenges outlined in Chapter~\ref{ch:mic2015}.}

\chapteritem{evaluation}{presents a method to evaluate the performance of binary classifiers without negative labels. This method enables estimating most commonly used performance metrics in a semi-supervised setting, which was uncharted territory.}

\chapteritem{diabetesjmlr}{integrates all machine learning aspects into a workflow to predict which patients are likely to start glucose-lowering pharmacotherapy, based exclusively on readily available, individual health expenditure records. This chapter combines all techniques described in previous chapters.}

\chapteritem{conclusion}{summarizes our work and describes potential use-cases and promising future research avenues. Finally, we conclude with some relevant tradeoffs from a policy perspective regarding the use of health and healthcare data for medical applications.}

\end{itemize}


\begin{subappendices}
\input{chapters/introduction/diabetes.tex}
\end{subappendices}


%%%%%%%%%%%%%%%%%%%%%%%%%%%%%%%%%%%%%%%%%%%%%%%%%%
% Keep the following \cleardoublepage at the end of this file, 
% otherwise \includeonly includes empty pages.
\cleardoublepage

% vim: tw=70 nocindent expandtab foldmethod=marker foldmarker={{{}{,}{}}}
