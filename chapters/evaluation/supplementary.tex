%%%%%%%%%%%%%%%%%%%%%%%%%%%%%%%%%%%%%%%%%%%%%%%%%%%%%%%%%%%%%%%%%%
%%%%%%%% ICML 2015 EXAMPLE LATEX SUBMISSION FILE %%%%%%%%%%%%%%%%%
%%%%%%%%%%%%%%%%%%%%%%%%%%%%%%%%%%%%%%%%%%%%%%%%%%%%%%%%%%%%%%%%%%

% Use the following line _only_ if you're still using LaTeX 2.09.
%\documentstyle[icml2015,epsf,natbib]{article}
% If you rely on Latex2e packages, like most moden people use this:
\documentclass{article}
\usepackage{fullpage}

% use Times
\usepackage{times}
% For figures
\usepackage{graphicx} % more modern
%\usepackage{epsfig} % less modern
\usepackage{subfigure} 
%\usepackage{subcaption} 

% For citations
%\usepackage{natbib}

% For algorithms
\usepackage{algorithm}
\usepackage{algorithmic}

% As of 2011, we use the hyperref package to produce hyperlinks in the
% resulting PDF.  If this breaks your system, please commend out the
% following usepackage line and replace \usepackage{icml2015} with
% \usepackage[nohyperref]{icml2015} above.
\usepackage[colorlinks=false,allbordercolors={1 1 1}]{hyperref}

% Packages hyperref and algorithmic misbehave sometimes.  We can fix
% this with the following command.
\newcommand{\theHalgorithm}{\arabic{algorithm}}

% Employ the following version of the ``usepackage'' statement for
% submitting the draft version of the paper for review.  This will set
% the note in the first column to ``Under review.  Do not distribute.''
%\usepackage{icml2015} 

% Employ this version of the ``usepackage'' statement after the paper has
% been accepted, when creating the final version.  This will set the
% note in the first column to ``Proceedings of the...''
%\usepackage[accepted]{icml2015}

\usepackage{amsthm}
\usepackage{amsmath}
\usepackage{amssymb}
\usepackage{tikz}
\usepackage{pgfplots}
\usepackage{ifpdf}
\usepackage{multirow}
\usepackage{booktabs}
\usepackage{color}
\usepackage{xspace}
\usepackage{mathtools}
\usepackage{fancyref}
\usepackage{multicol}

\input{definitions.tex}
%\usepackage{subcaption}

\usetikzlibrary{intersections}
\usetikzlibrary{patterns}
\usetikzlibrary{decorations.text}
\usetikzlibrary{decorations.markings}

% The \icmltitle you define below is probably too long as a header.
% Therefore, a short form for the running title is supplied here:
%\icmltitlerunning{Supplementary Materials}
\title{Supplementary Materials}
\date{}

\newcommand{\plotfigures}[2]{
\begin{figure*}[!h]
\centering
\subfigure[Rank CDF.]{\includegraphics[width=0.26\textwidth]{cdf_#1.pdf}}\qquad
\subfigure[ROC curve.]{\includegraphics[width=0.26\textwidth]{roc_#1.pdf}}\qquad
\subfigure[PR curve.]{\includegraphics[width=0.26\textwidth]{pr_#1.pdf}}\\
\caption{#2}
\label{figures-#1}
\end{figure*}
}

\newcommand{\covtype}{\textsc{covtype}\xspace}
\newcommand{\sensit}{\textsc{sensit}\xspace}
\addtolength{\parskip}{1mm}
\setlength{\parindent}{0mm}
\begin{document} 

\maketitle

\section{Content}
This file contains additional figures to demonstrate the efficacy of our proposed approach. All figures included here are generated using the Python scripts and data provided in the supplementary materials (Section~\ref{scripts} contains more details). In the main paper we have shown figures in which we retained $5\%$ of known positive labels for both data sets (Figures~8 to 10). In this document, we show results in which we retain only $20\%$, $10\%$ and $1\%$ of positive labels for \sensit in Figures~\ref{figures-sensit_2_semi_resvm_20percent} to \ref{figures-sensit_2_semi_resvm_1percent} and for \covtype in Figures~\ref{figures-covtype_pu_resvm_20percent} to \ref{figures-covtype_pu_resvm_1percent}. We used $95\%$ confidence intervals estimated via the bootstrap as in the main paper. The method can be tested under different circumstances via the provided scripts.

The results show that the estimates are consistently good, even for very low numbers of known positives (as little as 52 in Figure~\ref{figures-sensit_2_semi_resvm_1percent}). The effect of the number of known positives is particularly dramatic in PR curves, hence we recommend using ROC curves based on our estimated bounds. Incorrectly assuming $\hat{\beta}=0$ can render estimated PR curves useless.

The bounds on the ROC curve in Figure~\ref{figures-sensit_2_semi_resvm_1percent} exhibit a shape which is impossible for real curves to follow, though this is not an implementation error. This apparent jitter is caused by finite sample effects, specifically jumps in the rank CDF of known positives and the confidence intervals that are based on it. It is important to note that these bounds are computed based on contingency tables of sets of surrogate positives. These sets of surrogate positives are redefined for each rank, which does not happen when computing an ROC curve in the usual way (e.g. the set of ground truth positives is constant under normal circumstances, while our method infers surrogate positives for every rank based on the given bounds on rank CDF). This illustrates the limitation of our approach when the number of known positives is very low (a direct result of poor estimates of rank CDF).

\subsection{Python scripts and data} \label{scripts}
All figures can be generated using the \verb+generate_figures.py+ script included in the supplementary materials, which in turn uses \verb+semisup_metrics.py+ (implementation of the estimation procedure) and \verb+data.pkl+ (rankings and decision values for both data sets). This script has been developed in Python 2.7 and tested in 2.7 and 3.4. It requires \textsc{Optunity} (\url{http://www.optunity.net}) and \textsc{matplotlib} (\url{http://www.matplotlib.org}) to be installed along with their dependencies. Typically, installation is quite simple using the following commands:
\begin{verbatim}
pip install optunity
pip install matplotlib
\end{verbatim}
For more detailed installation instructions please refer to the manuals of these respective packages. After installing all dependencies, you can run the script using:
\begin{verbatim}
python generate_figures.py
\end{verbatim}

The script prompts you to configure various aspects of the estimation (defaults yield the figures in the paper):
\begin{enumerate}
  \setlength{\itemsep}{0pt}
  \setlength{\parskip}{0pt}
\item data set to use (\covtype or \sensit)
\item fraction of positives/negatives for which labels will be kept (randomly selected)
\item width of confidence intervals on rank CDF of known positives and the estimation method (bootstrap or DKW)
\item (optional) the number of bootstrap resamples to use
\end{enumerate}

\newpage
\section*{Figures for \sensit}
\plotfigures{sensit_2_semi_resvm_20percent}{Example curves on the \sensit data set with $20\%$ of positives labeled, i.e. $|\knownpos|=1,050$, $|\latentpos|=4,200$ and $\pfrac=\hat{\pfrac}\approx23\%$.}

\plotfigures{sensit_2_semi_resvm_10percent}{Example curves on the \sensit data set with $10\%$ of positives labeled, i.e. $|\knownpos|=525$, $|\latentpos|=4,725$ and $\pfrac=\hat{\pfrac}\approx24\%$.}

\plotfigures{sensit_2_semi_resvm_1percent}{Example curves on the \sensit data set with $1\%$ of positives labeled, i.e. $|\knownpos|=52$, $|\latentpos|=5,198$ and $\pfrac=\hat{\pfrac}\approx26\%$.}

\newpage
\section*{Figures for \covtype}
\plotfigures{covtype_pu_resvm_20percent}{Example curves on the \covtype data set with $20\%$ of positives labeled, i.e. $|\knownpos|=4,000$, $|\latentpos|=16,000$ and $\pfrac=\hat{\pfrac}\approx44\%$.}

\plotfigures{covtype_pu_resvm_10percent}{Example curves on the \covtype data set with $10\%$ of positives labeled, i.e. $|\knownpos|=2,000$, $|\latentpos|=18,000$ and $\pfrac=\hat{\pfrac}\approx47\%$.}

\plotfigures{covtype_pu_resvm_1percent}{Example curves on the \covtype data set with $1\%$ of positives labeled, i.e. $|\knownpos|=200$, $|\latentpos|=19,800$ and $\pfrac=\hat{\pfrac}\approx50\%$.}


\end{document}
