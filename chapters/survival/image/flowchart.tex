
% Flowcharting techniques for easy maintenance
% Author: Brent Longborough
\documentclass[x11names]{article}
%\documentclass{standalone}


\usepackage{xcolor}

\usepackage{geometry}
\geometry{
paperwidth=11.7cm,
paperheight=20cm,
margin=0cm
}


\usepackage{tikz}
\usetikzlibrary{shapes,arrows,chains}
%%%<
\usepackage{verbatim}
\usepackage[active,tightpage]{preview}
\PreviewEnvironment{tikzpicture}
\setlength\PreviewBorder{5mm}%
%%%>
\begin{comment}
:Title: Easy-maintenance flowchart
:Tags: flowcharts
:Author: Brent Longborough
:Slug: flexible-flow-chart

  This TikZ example illustrates a number of techniques for making TikZ
  flowcharts easier to maintain:
    * Use of <on chain> and <on grid> to simplify positioning
    * Use of global <node distance> options to eliminate the need to 
      specify individual inter-node distances
    * Use of <join> to reduce the need for references to node names
    * Use of <join by> styles to tailor specific connectors
    * Use of <coordinate> nodes to provide consistent layout for
      parallel flow lines
    * A method for consistent annotation of decision box exits
    * A technique for marking coordinate nodes (for layout debugging)

    I encourage you to tinker at this file - add intermediate boxes,
    alter the global distance settings, and so on, to see how well (or
    ill!) it adapts.
\end{comment}
\begin{document}\sloppy
% =================================================
% Set up a few colours
% -------------------------------------------------
% Set up a new layer for the debugging marks, and make sure it is on
% top
\pgfdeclarelayer{marx}
\pgfsetlayers{main,marx}
% A macro for marking coordinates (specific to the coordinate naming
% scheme used here). Swap the following 2 definitions to deactivate
% marks.
\providecommand{\cmark}[2][]{%
  \begin{pgfonlayer}{marx}
    \node [nmark] at (c#2#1) {#2};
  \end{pgfonlayer}{marx}
  } 
\providecommand{\cmark}[2][]{\relax} 
% -------------------------------------------------
% Start the picture
\begin{tikzpicture}[%
%    >=triangle 60,              % Nice arrows; your taste may be different
    >=angle 60,
    start chain=going below,    % General flow is top-to-bottom
    node distance=7mm and 68mm, % Global setup of box spacing
    every join/.style={norm},   % Default linetype for connecting boxes
    shorten <= 1mm, shorten >= 1mm,
    ]
% ------------------------------------------------- 
% A few box styles 
% <on chain> *and* <on grid> reduce the need for manual relative
% positioning of nodes
\tikzset{
  base/.style={draw, on chain, on grid, align=center, minimum height=4ex},
  proc/.style={base, rectangle, align=center, text width=16em},
  test/.style={base, diamond, aspect=2, text width=5em},
  term/.style={proc, rounded corners, it, densely dotted},
  % coord node style is used for placing corners of connecting lines
  coord/.style={coordinate, on chain, on grid, node distance=6mm and 25mm},
  % nmark node style is used for coordinate debugging marks
  nmark/.style={draw, cyan, circle, font={\sffamily\bfseries}},
  % -------------------------------------------------
  % Connector line styles for different parts of the diagram
  norm/.style={->, draw},
  it/.style={font={\small\itshape}}
}
% -------------------------------------------------
% Start by placing the nodes
\node [proc] (p0) {NACM member population\\ on January 1, 2003 \\ ($n=4,567,172$)};

\node [term, join] (q_history) {patient was member before 2000?};
\node [proc, join] (p_history) {patients with known recent history \\ ($n=4,266,401$)};
\path (q_history) to node [xshift=1.5em] {$yes$} (p_history);

\node [term, join] (q_therapy_ever) {records of any antidiabetes therapy\\ up to and including 2013?};
\node [proc, join] (p_therapy_ever) {patients with known history\\ of antidiabetes therapy\\ ($n=490,052$)};
\path (q_therapy_ever) to node [xshift=1.5em] {$yes$} (p_therapy_ever);

%\node [term, join] (q_therapy_incl) {new or modified antidiabetes therapy\\ in selection period 2003--2007?};
%\node [proc, join] (p_therapy_incl) {patients that may be included\\ into one or more study groups\\ ($n=TODO$)};
%\path (q_therapy_incl) to node [xshift=1.5em] {$yes$} (p_therapy_incl);


\node [term, join] (q_additional) {does the patient satisfy\\ all final inclusion criteria?};
\node [proc, join] (p_additional) {patients eligible for\\assignment to study groups\\ ($n=115,896$)};
\path (q_additional) to node [xshift=1.5em] {$yes$} (p_additional);

\node [term, join] (q_priors) {records of antidiabetes therapy\\ prior to inclusion event?};
\node [proc, join] (p_priors) {patient can enter appropriate\\ \textbf{combination study group}\\ ($n=28,708$)};
\path (q_priors) to node [xshift=1.5em] {$yes$} (p_priors);

\node [proc, right=of q_history] (p_nohistory) {patients with insufficient \\historical records are \textbf{excluded} because we may be unaware of recent cardiovascular events and/or antidiabetes therapy};
\path (q_history) to node [yshift=1em] {$no$} (p_nohistory);
    \draw [->] (q_history) -- (p_nohistory);

\node [proc, right=of q_therapy_ever] (p_notherapy) {patient enters reference population\\ from which \textbf{controls} are sampled\\ ($n=3,776,349$)};
\path (q_therapy_ever) to node [yshift=1em] {$no$} (p_notherapy);
    \draw [->] (q_therapy_ever) -- (p_notherapy);
    
%\node [proc, right=of q_therapy_incl] (p_noincl) {ineligible for inclusion in any\\ study or control group\\ ($n=TODO$)};
%\path (q_therapy_incl) to node [yshift=1em] {$no$} (p_noincl);
%    \draw [->] (q_therapy_incl) -- (p_noincl);

\node [term, right=of q_additional, align=left] (q_additional_full) {final inclusion criteria:
\begin{itemize}
\item therapy must start/change in 2003--2007
\item patient must survive $\geq$ 30 days% after inclusion
\item same therapy for $\geq$ 90 days
\item age at inclusion must be $\geq$ 18
\end{itemize}
    };
    \draw [densely dotted, shorten <= 0mm, shorten >= 0mm] (q_additional) -- (q_additional_full);

\node [proc, right=of q_priors] (p_nopriors) {patient can enter appropriate\\ \textbf{mono} -or \textbf{combination\\ therapy study group} \\ ($n=87,188$)};
\path (q_priors) to node [yshift=1em] {$no$} (p_nopriors);
    \draw [->] (q_priors) -- (p_nopriors);

%\node [right=of p_priors] (tag_node) {};
\coordinate [right=of p_priors, below=9.5mm of p_nopriors] (tag_node);
\draw [dashed, thick, shorten <= 1mm, shorten >= 0mm]
  (p_nopriors.south) -- (tag_node.north);
\draw [dashed, thick, shorten <= 1mm, shorten >= 0mm]
  (p_priors.east) -- (tag_node.west);

\draw [->,dashed, thick, shorten >=1mm, shorten <= 0mm]
  (tag_node) -- ++(33mm,0) 
  |- node [near end, yshift=0.75em, it]
    {(Patient may enter additional study groups)} (p_therapy_ever);

% -------------------------------------------------
\end{tikzpicture}
% =================================================
\end{document}
