\chapter*{Beknopte samenvatting}

Diabetes mellitus is een metabolische stoornis die gekarakteriseerd wordt door chronische hyperglycemie, hetgeen zware schade kan veroorzaken aan verschillende biologische systemen. Diabetes is een dodelijke pandemie die leidt tot een enorme belasting op wereldwijde gezondheidszorg. De impact van diabetes zal verder toenemen in de komende jaren aangezien de globale prevalentie nog steeds stijgt, in het bijzonder deze van type 2 diabetes. In ontwikkelde landen is de stijgende prevalentie voornamelijk te wijten aan vergrijzing, veranderingen in levensstijl en langere overleving van diabetespati\"enten. Wanneer diabetes vroeg gedetecteerd wordt kan de ziekte goed behandeld worden, maar vroegtijdige detectie blijkt problematisch aangezien de periode tussen de ontwikkeling en diagnose van diabetes verschillende jaren kan duren en \'e\'en derde van de patienten naar schatting niet gediagnosticeerd zijn.

%Diabetes mellitus is a metabolic disorder characterized by chronic hyperglycemia, which may cause serious harm to many of the body's systems. Diabetes is a deadly pandemic which presents a significant burden on healthcare systems worldwide, and will continue to do so as its global prevalence rises rapidly (particularly type 2 diabetes). In developed countries, the rising prevalence is primarily driven by population aging, lifestyle changes and greater longevity of diabetes patients. Diabetes can be managed effectively when detected early, but this proves difficult as the time between onset and clinical diagnosis may span several years and over one third of patients are undiagnosed.

Wij hebben het potentieel onderzocht om een kosteneffectieve, nationale screening-methode voor (type 2) diabetes mellitus to ontwikkelen op basis van Belgische ziekenfondsgegevens. Dit zou een meerwaarde kunnen betekenen in secundaire preventie als we hiermee sneller pati\"enten kunnen diagnosticeren en vervolgens behandelen voor de ziekte onherroepelijke schade heeft aangericht. We maakten gebruik van ziekenfondsgegevens die verzameld werden door de Landsbond der Christelijke Mutualiteiten (CM) -- het grootste ziekenfonds in Belgi\"e. Deze data omvat simpele biografische informatie en records van alle terugbetaalde medische interventies en aankopen van medicijnen, wat in zijn geheel een longitudinaal overzicht over lange termijn geeft van de medische uitgaven van de meer dan 4 miljoen leden van de CM.

%We investigated the potential of Belgian health expenditure data as a basis to build a cost-effective population-wide screening approach for (type 2) diabetes mellitus, aspiring to improve secondary prevention by speeding up the diagnosis of patients in order to initiate treatment before the disease has caused irrevocable damage. We used health expenditure data collected by the National Alliance of Christian Mutualities -- the largest social health insurer in Belgium. This data comprises basic biographic information and records of all refunded medical interventions and drug purchases, thus providing a long-term longitudinal overview of over 4 million individuals' medical expenditure histories.

Screening werd geformuleerd als een binaire klassificatie-taak, waarin diabetespati\"enten de positieve klasse voorstellen. Door beperkingen van ziekenfondsgegevens konden we geen verzameling van gekende negatieven bekomen (dit zijn mensen die zeker geen diabetes hebben). Daarom hebben we modellen moeten opstellen op basis van positieve en niet-gelabelde data. Tijdens dit project hebben we twee contributies gedaan tot dit subdomein van semi-supervised learning: (i) een nieuwe leermethode die robuust is tegen valse positieven en (ii) een aanpak om de performantie van modellen te evalueren via traditionele metrieken zonder gekende negatieven in de test set. Verder hebben we de overleving van pati\"enten die startten met verscheidene glucoseverlagende farmacotherapi\"een in kaart gebracht en twee open source pakketten ontwikkeld voor machine learning: \'e\'en voor ensemble learning en \'e\'en om hyperparameter-optimalisatie te automatiseren.

%Screening was formulated as a binary classification task, in which diabetes patients represent the positive class. Due to limitations of health expenditure data, we were unable to identify a set of known negatives. Hence, we had to learn classifiers from positive and unlabeled data. During this project we made two contributions to this subdomain of semi-supervised learning: (i) a novel learning method which is robust to false positives and (ii) an approach to evaluate classifiers using traditional metrics without known negatives in the test set. Additionally, we mapped the survival of patients starting various antidiabetic pharmacotherapies and developed two open-source machine learning packages: one for ensemble learning and another to automate hyperparameter search.

We hebben een screening-methode ontwikkeld die qua performantie competitief is met de beste, bestaande alternatieven. Dit overtrof onze verwachtingen, aangezien ziekenfondsgegevens weinig tot geen informatie bevatten over een aantal typische risicofactoren die door andere screening-methodes gebruikt worden (BMI, levensstijl, genetische aanleg, \ldots). Hieruit volgt dat de combinatie van ziekenfondsgegevens en bijkomende informatie over risicofactoren een interessante piste is voor toekomstig onderzoek in screening voor diabetes mellitus. Tenslotte heeft onze aanpak een zeer lage operationele kost omdat de methode volledig gebaseerd is op gegevens die reeds ter beschikking staan, hetgeen een oplossing biedt aan \'e\'en van de belangrijke barri\`eres voor nationale screening-methodes voor diabetes.

%We built a screening method with competitive performance to existing state-of-the-art approaches. This exceeded our expectations, since health expenditure data omits most info about the typical risk factors used by other screening methods (BMI, lifestyle, genetic predisposition, \ldots). As such, the combination of health expenditure data and other risk factors is a promising avenue for future research in screening for diabetes mellitus. Finally, our approach has a very low operational cost since we only used readily-available data, which effectively removes one of the key barriers of population-wide screening for diabetes.


%%%%%%%%%%%%%%%%%%%%%%%%%%%%%%%%%%%%%%%%%%%%%%%%%%
% Keep the following \cleardoublepage at the end of this file, 
% otherwise \includeonly includes empty pages.
\cleardoublepage

% vim: tw=70 nocindent expandtab foldmethod=marker foldmarker={{{}{,}{}}}
